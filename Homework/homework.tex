\documentclass[12pt]{article}

\input{homeworkPream}

\begin{document}

%----------BEGIN DATA----------

% table of data for plotting
%\pgfplotstableread{
%Amplitude Amplitude-min Amplitude-max Period Period-min Period-max
%5         7             3             1.932  1.927      1.937
%17        15            19            1.94   1.93       1.95
%25        23            27            1.96   1.95       1.97
%40        36            44            2.01   2.00       2.02
%53        49            57            2.04   2.03       2.05
%67        61            73            2.12   2.10       2.14
%}{\219Data}

%-----------END DATA-----------

%----------BEGIN HEADER----------

% right justify header info
\hspace*{\fill} Ryan Wojtyla \\
\hspace*{\fill} 25 September 2018 \\

\begin{center}
  {\Large Optics Lab Homework} \\
\end{center}

%-----------END HEADER-----------

%----------BEGIN 2.11----------

\section*{2.11}

To subtract two values with uncertainty, the values themselves are first
subtracted, then their uncertainties are added together. For the first row of
the chart:

\[ (3.0 \pm 0.3) - (2.7 \pm 0.6) \]
\[ (3.0 - 2.7) \pm (0.3 + 0.6) \]

\begin{tcolorbox}[title=The first row of the new \( L - L^\prime \) column.]
  \[ 0.3 \pm 0.9 \]
\end{tcolorbox}

The remainder of the values were generated by feeding the data from the table
into the following Haskell program:

\lstset{numbers=left,
        language=Haskell,
        basicstyle=\ttfamily,
        keywordstyle=\color{blue}\ttfamily,
        stringstyle=\color{green}\ttfamily,
        commentstyle=\color{red}\ttfamily
      }
\lstinputlisting{2.11/subtractUncertainties.hs}

The table can then be completed:

\begin{figure}[H]
  \begin{center}
    \label{tab:2.11}
    \begin{tabular}{|c|c|c|}
      \hline
      Initial \(L\) & Final \(L^\prime\) & \(L - L^\prime\) \\
      \hline
      \(3.0 \pm 0.3\) & \(2.7 \pm 0.6\) & \(0.3 \pm 0.9\) \\
      \(7.4 \pm 0.5\) & \(8.0 \pm 1\) & \(-0.6 \pm 1.5\) \\
      \(14.3 \pm 1\) & \(16.5 \pm 1\) & \(-2.2 \pm 2.0\) \\
      \(25 \pm 2\) & \(24 \pm 2\) & \(1.0 \pm 4.0\) \\
      \(32 \pm 2\) & \(31 \pm 2\) & \(1.0 \pm 4.0\) \\
      \(37 \pm 2\) & \(41 \pm 2\) & \(-4.0 \pm 4.0\) \\
      \hline
    \end{tabular}
    \caption{The completed table for problem 2.11, with included third column.}
  \end{center}
\end{figure}

All results, except for the one in the third row, are consistent with the
conservation of angular momentum, because the uncertainties allow the value to
cross zero. In the case of the third row, the closest the uncertainty brings the
value to zero is -0.2.

%-----------END 2.11-----------

%----------BEGIN 2.19----------


%\begin{figure}[H]
%  \begin{center}
%    \begin{tikzpicture}[scale=1.0]
%      \begin{axis} [xlabel=Amplitude A (\SI{degree}), ylabel=Period T
%        (\SI{second})]
%        \addplot
%[error bars/.cd, x dir=plus, x explicit] table 
%[x index=0, y index=1, y error index=2]
%{2.19/table.csv};
%        \addplot
%        [error bars/.cd, x dir=plus, x explicit]
%        table[x=Amplitude, y=Period, x error expr=\thisrow{xErrMax}]
%        {2.19/table.csv}
%        \addplot
%        [error bars/.cd, y dir=plus, y explicit]
%        table[x=Amplitude, y=Period, y error expr=\thisrow{yErrMin}]
%        {2.19/table.csv}
%        \addplot
%        [error bars/.cd, y dir=plus, y explicit]
%        table[x=Amplitude, y=Period, y error expr=\thisrow{yErrMax}]
%        {2.19/table.csv}

%-----------END 2.19-----------

%----------BEGIN 3.26----------

\section*{3.26}

\subsection*{(a)}

Since \( \mu = 0.10 \pm 0.01 \si{\centi\meter\squared\per\gram} \), it can be
determined from the graph that \( E = 0.7 \pm \delta E\), where \( \delta E
= E(\mu_{best} + \delta \mu) - E(\mu_{best}) \):

\begin{align*}
  E ={}& 0.7 \pm (|E(\mu_{best} + \delta \mu) - E(\mu_{best}|) \\
  E ={}& 0.7 \pm (|E((0.10) + (0.01)) - E((0.10))|)  \\
  E ={}& 0.7 \pm (|(0.65) - (0.7)|)  \\
\end{align*}

\begin{tcolorbox}
  \[ E = 0.7 \pm 0.05 MeV \nonumber \]
\end{tcolorbox}

\subsection*{(b)}

If \( \mu = 0.22 \pm 0.01 \si{\centi\meter\squared\per\gram} \), then \( E = 0.4 \pm
\delta E \):

\begin{align*}
  E ={}& 0.4 \pm (|E(\mu_{best} + \delta \mu) - E(\mu_{best}|) \\
  E ={}& 0.4 \pm (|E((0.22) + (0.01)) - E((0.22))|)  \\
  E ={}& 0.4 \pm (|(0.49) - (0.5)|)  \\
\end{align*}

\begin{tcolorbox}
  \[ E = 0.4 \pm 0.01 MeV \nonumber \]
\end{tcolorbox}

%-----------END 3.26-----------

%----------BEGIN 3.31----------

\section*{3.31}

\subsection*{(a)}

The uncertainty (\( \delta q \)) in any function of one variable (\( q(x) \))
can be expressed as: \( \delta q = \abs{\frac{dq}{dx}} \delta x \). If \( q(x) =
\sin{(\theta)} \) and \( \theta = 125 \pm 2 \si{degree} \):

\begin{align*}
  \delta sin{(\theta)} ={}& \abs{\frac{d(\sin{(\theta)})}{d \theta}} \delta
                               \theta \\
  \delta sin{(\theta)} ={}& \abs{\cos{(\theta)}} \delta \theta \\
  \delta sin{(125)} ={}& \abs{\cos{(125)}} (2) \\
\end{align*} 

\begin{tcolorbox}[title=The uncertainty of \( \sin{(125)} \).]
  \[ \delta \sin{(125)} = 1.15 \si{degree} \]
\end{tcolorbox}
j
\begin{tcolorbox}[title=The value of \( \sin{(\theta)} \) where 
  \( \theta = 125 \pm 2 \si{degree} \).]
  \[ \sin{(125)} = 0.819 \pm 1.15 \si{degree} \]
\end{tcolorbox}

\subsection{(b)}



\subsection{(c)}



%-----------END 3.31-----------

%----------BEGIN 4.13----------



%-----------END 4.13-----------

%----------BEGIN 4.20----------



%-----------END 4.20-----------

%----------BEGIN 5.31----------



%-----------END 5.31-----------

%----------BEGIN 5.35----------



%-----------END 5.35-----------

%----------BEGIN 6.1----------



%-----------END 6.1-----------

%----------BEGIN 7.3----------



%-----------END 7.3-----------

%----------BEGIN 7.5----------



%-----------END 7.5-----------

%----------BEGIN 8.7----------



%-----------END 8.7-----------

%----------BEGIN 8.17----------



%-----------END 8.17-----------

\end{document}
