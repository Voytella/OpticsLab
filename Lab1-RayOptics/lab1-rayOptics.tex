\documentclass[12pt]{article}

\input{physicsPream}

\begin{document}

%----------BEGIN TITLE----------

PHY 4021: Experiments in Optics \\
Experiment 1: Ray Optics \\
Ryan Wojtyla \\
Akshath Wikramanayake \\
09 Oct 2018 \\


%-----------END TITLE-----------

\section{Experiments}

%----------BEGIN EXPERIMENT 1----------

\subsection{Introduction to Ray Optics}

\subsubsection{Straight Line Propagation of Light}

\subsubsubsection{}

While the rays are straight, they are not parallel with each other.

\subsubsubsection{}

As the rays' distance from the Slit Plate increases, their width increases while
their distinctness decreases.

\subsubsubsection{}

All of the rays appeared to originate from the Light Source, even when viewed
from a slight angle.

\subsubsubsection{}

As the angle of the Slit Plate increases, becomes more horizontal, the width of
the rays increases while their distinctness decreases.

\subsubsubsection{}

The images are most distinct when the Slit Plate is entirely vertical, and
they are least distinct when it is horizontal.

\subsubsubsection{}

<come back to this one>

\subsubsection{Ray Tracing: Locating the Filament}

\subsubsubsection{}

The distance between the reference mark in the center of the Ray Table and the
point of intersection of the rays at the filament
is \(d_e = 24.1 \pm 0.05 \si{\centi\meter}\).

\subsubsubsection{}

The distance between the filament and the center of the Ray Table was measured
to be \(d_t = 25.6 \pm 0.05 \si{\centi\meter}\).

\subsubsubsection{}

The two measurements have a percent error, where
\(\%_{err} = \frac{|d_t - d_e|}{d_t} \cdot 100\%\), of \(\%_{err} = 5.86\%\), and a percent
uncertainty, where \(\delta \%_{err} = \frac{\%_{err}}{d_e}\), of \(\delta
\%_{err} = 0.249 \%\). Although the percent error, \(5.86\% \pm 0.249\%\), is low,
it is, nonetheless, present.

%-----------END EXPERIMENT 1-----------

%----------BEGIN EXPERIMENT 2----------

\subsection{The Law of Reflection}

\subsubsection{Data}



\subsubsection{Questions}

\subsubsubsection{}

The results of the two trials are the same.

\subsubsubsection{}

The incident ray, reflected ray, and normal all lie on the same plane because
the reflected and incident rays are visible on the 2D surface of the Ray
Table. Because they are both visible on the 2D surface, they must both reside in
the same 2D plane.

\subsubsubsection{}

The angle of incidence and the angle of reflection are both the same value. 

\subsubsection{Data}

\begin{figure}[H]
  \label{tab:2.1}
  \begin{center}
    \begin{tabular}{|c|c|c|}
      \hline
      Incidence (\si{degree}) & Reflection\(_1\) (\si{degree}) & Reflection\(_2\)
                                                             (\si{degree}) \\
      \hline
      0 & 0 & 0 \\
      10 & 10 & 10 \\
      20 & 20 & 20 \\
      30 & 30 & 30 \\
      40 & 40 & 40 \\
      50 & 50 & 50 \\
      60 & 60 & 60 \\
      70 & 70 & 70 \\
      80 & 80 & 80 \\
      90 & 90 & 90 \\
      \hline
    \end{tabular}
  \end{center}
  \caption{\textbf{Table 2.1:} The two symmetric angles of reflection resulting
    from the indicated angle of incidence.}
\end{figure}
 
%-----------END EXPERIMENT 2-----------

%----------BEGIN EXPERIMENT 3----------

\subsection{Image Formation in a Plane Mirror}

\subsubsection{Sketch}

<include pictures>

\subsubsection{Questions}

\subsubsubsection{}

The rays do seem to follow a straight line into the mirror.

\subsubsubsection{}

The distance from the filament to the plane of the mirror (\(d_1\)) was measured
to be \(30.1 \pm 0.05 \si{\centi\meter}\).

\subsubsubsection{}

The perpendicular distance from the image of the filament to the plane mirror
(\(d_2\)) was measured to be \(30.0 \pm 0.05 \si{\centi\meter}\).

\subsubsubsection{}

The image will always appear to be the same distance away as if it were being
viewed straight on without a mirror. ???

%-----------END EXPERIMENT 3-----------

%----------BEGIN EXPERIMENT 4----------

\subsection{The Law of Refraction}

\subsubsection{Data}

\begin{figure}[H]
  \label{tab:4.1}
  \begin{center}
    \begin{tabular}{|c|c|c|}
      \hline
      Incidence (\si{degree}) & Refraction\(_1\) (\si{degree}) &
                                                                 Refraction\(_2\)
                                                                 (\si{degree})
      \\
      \hline
      0  & 0    & 0 \\
      10 & 7    & 6 \\
      20 & 13   & 13 \\
      30 & 20   & 20 \\
      40 & 25   & 26 \\
      50 & 31.5 & 31.5 \\
      60 & 36   & 36 \\
      70 & 41.5 & 42 \\
      80 & n/a  & n/a \\
      90 & n/a  & n/a \\
      \hline
      \end{tabular}
      \end{center}
      \caption{\textbf{Table 4.1:} The two symmetric angles of refraction from each
  angle of incidence.}
\end{figure}

\subsubsection{Questions}

\subsubsubsection{}

The ray is not bent when it passes into the lens perpendicular to the lens' flat
surface.

\subsubsubsection{}

The ray is slightly bent when it passes out of the lens perpendicular to the
lens' curved surface.

\subsubsubsection{}

While the two sets of measurements are very nearly identical, there are minor
differences between the two. These differences may be attributed to our
inability to perfectly align the ray with the Ray Table.

%-----------END EXPERIMENT 4-----------

%----------BEGIN EXPERIMENT 5----------



%-----------END EXPERIMENT 5-----------

%----------BEGIN EXPERIMENT 6----------



%-----------END EXPERIMENT 6-----------

%----------BEGIN EXPERIMENT 7----------



%-----------END EXPERIMENT 7-----------

%----------BEGIN EXPERIMENT 8----------



%-----------END EXPERIMENT 8-----------

%----------BEGIN EXPERIMENT 9----------



%-----------END EXPERIMENT 9-----------

%----------BEGIN EXPERIMENT 10----------



%-----------END EXPERIMENT 10-----------

%----------BEGIN EXPERIMENT 11----------



%-----------END EXPERIMENT 11-----------

%----------BEGIN EXPERIMENT 13----------



%-----------END EXPERIMENT 13-----------

%----------BEGIN EXPERIMENT 14----------



%-----------END EXPERIMENT 14-----------

%----------BEGIN EXPERIMENT 15----------



%-----------END EXPERIMENT 15-----------

%----------BEGIN EXPERIMENT 16----------



%-----------END EXPERIMENT 16-----------

%----------BEGIN EXPERIMENT 18----------



%-----------END EXPERIMENT 18-----------

%----------BEGIN EXPERIMENT 19----------



%-----------END EXPERIMENT 19-----------

%----------BEGIN EXPERIMENT 20----------



%-----------END EXPERIMENT 20-----------

%----------BEGIN EXPERIMENT 21----------



%-----------END EXPERIMENT 21-----------

%----------BEGIN EXPERIMENT 22----------



%-----------END EXPERIMENT 22-----------

%----------BEGIN CONCLUSION----------



%-----------END CONCLUSION-----------

\end{document}
