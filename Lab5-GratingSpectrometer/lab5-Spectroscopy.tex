\documentclass{article}

\input{physicsPream}

\begin{document}

%----------BEGIN TITLEPAGE----------

\begin{titlepage}

  \title{Lab 5: Spectroscopy}
  \author{Ryan Wojtyla \\
    \textbf{Partners:} \\
    Keefe Kamp \\
    Jacquelyne Miksanek \\
    Akshath Wikramanayake \\
    }
  \date{November 20, 2018}

  \maketitle

  \begin{center}
    Abstract
  \end{center}

  \qq

  \thispagestyle{empty}

\end{titlepage}

%-----------END TITLEPAGE-----------

\section{Experiment 1: Emission Spectrum}

%----------BEGIN EXPERIMENT 1----------

\subsection{Objective}

\qq The purpose of this experiment was to determine the wavelengths of the
colors present in the emission spectrum of mercury vapor. This was accomplished
by shining a beam emitted by mercury vapor through a diffraction grating that
was then rotated. The angles at which each color revealed itself was recorded,
and that angle was used to calculate the wavelength of the colors.

\subsection{Theory}

\qq All the atoms of a particular element have a certain set of discrete
energy levels to which atoms' electrons can be excited. When the electrons fall
back down to their normal positions, a photon with energy equal to one of the
energy levels is emitted. An element's set of energy levels is referred to as
its emission spectrum. 

\qq The energy contained within these energy levels is determined with \(E =
\frac{hc}{\lambda}\). Since \(hc\) is a constant, the energy is dependent solely
upon the wavelength, \(\lambda\), which is the value we are finding in this
experiment. While the wavelength of a particular color cannot be measured
directly, it can be found with

\begin{equation}
  \label{eqn:wavelength}
  \lambda = d \sin{(\lambda)}
\end{equation}

where \(d\) is the distance between the lines of the diffraction grating and
\(\theta\) is the angle of diffraction for a color.

\subsection{Equipment}

\qq

\subsection{Procedures}

\qq

\subsection{Data and Analysis}

\qq

\subsection{Conclusion}

\qq

%-----------END EXPERIMENT 1-----------

\section{Experiment 2: Absorption Spectrum}

%----------BEGIN EXPERIMENT 2----------

\subsection{Objective}

\qq

\subsection{Theory}

\qq

\subsection{Equipment}

\qq

\subsection{Procedures}

\qq

\subsection{Data and Analysis}

\qq

\subsection{Conclusion}

\qq

%-----------END EXPERIMENT 2-----------

\end{document}
